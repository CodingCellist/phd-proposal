\documentclass[11pt]{article}
\usepackage[utf8]{inputenc}
\usepackage[a4paper, total={6in, 8in}]{geometry}
\usepackage{indentfirst}
\usepackage{idrislang}

\usepackage[backend=biber, maxnames=2, style=numeric, giveninits=true, sortcites=true]{biblatex}
\DeclareNameAlias{author}{family-given}
\addbibresource{lsv-types.bib}

\title{What's my type?\\Examining how model checking and type systems may
       complement and help each other}
\date{}

\begin{document}
    \maketitle

    \section*{Background}
    There has been an explosion in where computer systems are found. They are
    everywhere and they control more and more of our lives, directly or
    indirectly. A couple of examples of this are power plants, planes, sewage
    systems, medical devices, and spacecrafts. All of these systems use a
    computer at some point, usually to control a part of them. The systems
    given as examples are all critical in the sense that a failure would be
    catastrophical, potentially resulting in the loss of lives.
    
    For certain critical systems, model checking is a way of guaranteeing that
    the software works as expected by exhaustively testing every possible state
    the modelled program or protocol could be in. However, model checking is
    also slow and often runs into the ``State Explosion Problem''
    \cite{goos_state_1998} where the model to be checked results in many more
    states than could physically be stored, even on modern hardware. This
    problem has been, and still is, studied in great detail
    \cite{goos_progress_2001,stuart_simulation-verification_2001,demri_parametric_2006,clarke_model_2008,kress-gazit_correct_2011,meyer_model_2012}
    but the most common solution remains to simplify the actual model into
    smaller submodels (e.g. \cite{yan_sun_verifying_2007}) which can result in
    problems going unnoticed due to it only emerging when the system as a whole
    is considered.\\
    
    Type systems allow us to formulate the inputs and outputs of programs in a
    way that the compiler can then check that we are indeed using the right
    things. This allows us to be confident that the program we have written is
    correct with respect to the defined types. A simple example (from
    \cite{brady_type-driven_2017}) models a door that can be opened, closed, and
    where the doorbell can only be rung if the door is closed:
    \begin{idrislisting}
data DoorState = DoorClosed | DoorOpen

data DoorCmd : Type -> DoorState -> DoorState -> Type
  where
    Open : DoorCmd     () DoorClosed DoorOpen
    Close : DoorCmd    () DoorOpen DoorClosed
    RingBell : DoorCmd () DoorClosed DoorClosed
    
    Pure : ty -> DoorCmd ty state state
    (>>=) : DoorCmd a state1 state2 ->
            (a -> DoorCmd b state2 state3) ->
            DoorCmd b state1 state3
    \end{idrislisting}
    Given the types and actions described above, a program which uses the types
    and type-checks must correctly follow the `protocol' of operating the door.
    The advantage to using types and type-checking is that useful programs can
    be modelled and type-checked (proven to be valid) \textit{without}
    running into the state explosion problem.
    
    However, it is not known whether the types correctly model the problem we
    are trying to solve. The programs may type-check but if the types themselves
    do not capture the problem correctly, then that is never caught and the
    programs are still invalid (despite the type-checker not knowing so).
    
    \section*{Proposal}
    The aim of this research project would be to investigate how formal methods
    and model checking could complement or assist dependent types and
    functional programming and vice versa.
    
    As discussed in the background section above, there is no guarantee that the
    types correctly model the desired problem. Given that model checking can be
    used for checking that protocols are valid (e.g. that it is possible to
    reach a certain state under the current protocol), it is possible that model
    checking could also be used to answer the question ``Do my types correctly
    model my problem?'' The door example could be modelled in \textsc{Spin} with
    the following Promela code:
    \begin{code}
mtype:state = { DoorOpen, DoorClosed };
mtype:action = { Open, Close, RingBell };

chan c = [0] of { mtype:action };

mtype:state doorstate;

init
{
    do
    :: doorstate = DoorOpen; break;
    :: doorstate = DoorClosed; break;
    od
    run SENDER();
    run RECEIVER();
}

proctype SENDER()
{
    do
    :: c!Open
    :: c!Close
    :: c!RingBell
    od
}

proctype RECEIVER()
{
    do
    :: (c?Close && doorstate == DoorOpen) ->
           doorstate = DoorClose;
    :: (c?Open && doorstate == DoorClosed) ->
           doorstate = DoorOpen;
    :: (c?RingBell && doorstate == DoorClosed) ->
           doorstate = DoorClose;
    od
}
    \end{code}
    With this model in Promela, we could then use Linear Temporal Logic and
    \textsc{Spin} to examine questions like ``Is it always possible to open the
    door?'' or ``Can the door infinitely often be in the \texttt{DoorOpen}
    state?''\\
    
    Initially, an investigation could be conducted into how type checking
    dependent types and model checking might overlap. For example: for
    verifying that the index of a fixed-length array does not go out of bounds,
    how does a dependent type system model and verify the indices, and how
    would this be done in a model checker like \textsc{Spin}?
    
    Based on the information obtained from this investigation, a framework for
    model checking types could be drafted, discussing potential compromises
    that may need to be made. This framework could then either be implemented
    as a formal model in an existing modelling language, e.g. \textsc{Promela},
    or it may prove necessary to build a custom model checker (or customise an
    existing one).
    
    Overall, the aim of the project would be to explore how model checking
    dependent types could strengthen (or even prove) that the types capture the
    scenario the user is trying to model correctly. Additionally, going the
    other way, the aim would also be to explore how dependent types can help
    guarantee that the programs written are correct, instead of having to
    rewrite (often in a significantly simplified form) and check them in a
    formal modelling language.
    
    \newpage
    
    \printbibliography
    
\end{document}
